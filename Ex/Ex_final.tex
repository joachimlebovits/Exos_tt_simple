\RequirePackage{comment}
%%begin_old_tag
%\begin{comment}%_1
\newif 
 
\documentclass[a4paper,11pt,svgnames,openany,english]{book}
%\renewcommand{\partname}{Partie}
\includecomment{english}
\excludecomment{french}
 
\input{/Users/jlebovits/Dropbox/5-Latex_et_tous_logiciels_de_math_et_terminal/1-Latex/2-Fichiers_Tex_a_inclure/Inputgene/Inputgene}
%\end{comment}%_1
%%end_old_tag


%%begin_new_tag
\begin{comment}%_2
\documentclass{book}
\usepackage[T1]{fontenc}
\usepackage[utf8]{inputenc}
\usepackage{amsmath,amssymb}
\usepackage{iftex}
\usepackage{pythontex}
\usepackage{lmodern}

\input{/Users/jlebovits/Dropbox/5-Latex_et_tous_logiciels_de_math_et_terminal/1-Latex/2-Fichiers_Tex_a_inclure/Inputgene/Inputgene_pr_pandoc}
\end{comment}%_2
%%end_new_tag










\begin{document}


%\EX

%{\bfseries

{\Large   Exercice
  Exercise  
}\num
%}:

\smallskip


 

\begin{comment}
$\N$
\end{comment}




%\begin{comment}
 
 
Let $A$ and $P$ be the two following matrices.
 
%\end{comment}




\begin{align}
%\label{eozrifj}
%\tag{pouetpouet}
&A:=  \left[\begin{matrix}1 & 2 & -2\\2 & 1 & -2\\2 & 2 & -3\end{matrix}\right]&
& \& & 
&P:=  \left[\begin{matrix}1 & 0 & 1\\1 & 1 & 1\\1 & 1 & 2\end{matrix}\right].&
\end{align}

% 
 
%According to \eqref{eozrifj}, Equality, we clearly have 
% 







 
 
Denote  $P^{-1}$, the matrix inverse of $P$, if it exists.
 

\ben

\item 
 
 
Determine the inverse of matrix $P$, denoted $P$.
 



\item
 
 
Define
 
$D=P^{-1}AP$. 
%
%
 
 
Compute $D, D^{2}, D^{3}$ and deduce $D^{n}$, for all $n$ in $\N$.
 


\item
 
 
From the results obtained in the previous questions, compute  $A^{n}$, for all $n$ in $\N$.
 


 


\een





\end{document}

