\RequirePackage{comment}
%%begin_old_tag
%\begin{comment}%_1
\newif\iffrancais
%commenter la ligne suivante pour l'anglais
\francaistrue %si c'est commenté alors ce sera de l'anglais. Si c'est décommenté alors tout sera en fran\c cais
%
\iffrancais
\documentclass[a4paper,11pt,svgnames,openany,french]{book}
\includecomment{french}
\excludecomment{english}
\else
\documentclass[a4paper,11pt,svgnames,openany,english]{book}
%\renewcommand{\partname}{Partie}
\includecomment{english}
\excludecomment{french}
\fi
\input{/Users/jlebovits/Dropbox/5-Latex_et_tous_logiciels_de_math_et_terminal/1-Latex/2-Fichiers_Tex_a_inclure/Inputgene/Inputgene}
%\end{comment}%_1
%%end_old_tag


%%begin_new_tag
\begin{comment}%_2
\documentclass{book}
\usepackage[T1]{fontenc}
\usepackage[utf8]{inputenc}
\usepackage{amsmath,amssymb}
\usepackage{iftex}
\usepackage{pythontex}
\usepackage{lmodern}

\input{/Users/jlebovits/Dropbox/5-Latex_et_tous_logiciels_de_math_et_terminal/1-Latex/2-Fichiers_Tex_a_inclure/Inputgene/Inputgene_pr_pandoc}
\end{comment}%_2
%%end_new_tag










\begin{document}


\newtheorem{Thm}{Theoreme}
%\newtheorem*{rem}{Remark}


\begin{Thm}
%\label{oeizjdozij}
This statement is true, I guess.
\begin{equation}
\label{proto}
 2x +3 = 2.
\end{equation}
\end{Thm}


On peut donc dire que le théorème \ref{oeizjdozij} vérifie l'égalité \eqref{proto}.


\begin{rem}
This statement is true, I guess.
\end{rem}



%\EX

%{\bfseries

{\Large \iffrancais Exercice
\else Exercise \fi
}\num
%}:

\smallskip


\begin{sympycode}
A= Matrix([[1,2,-2],[2,1,-2],[2,2,-3]]) 
P= Matrix([[1,0,1],[1,1,1],[1,1,2]])  
\end{sympycode}

\begin{comment}
$\N$
\end{comment}




%\begin{comment}
\iffrancais
Soit $A$ et $P$ les deux matrices définies ci-dessous.
\else
Let $A$ and $P$ be the two following matrices.
\fi
%\end{comment}




\begin{align}
&A:=  \sympy{A}&
& \& & 
&P:=  \sympy{P}.&
\end{align}

%\iffrancais
%Selon \eqref{eozrifj}, on a l'égalité
%\else
%According to \eqref{eozrifj}, Equality, we clearly have 
%\fi







\iffrancais
On note $P^{-1}$, l'inverse de la matrice $P$, si elle existe.
\else
Denote  $P^{-1}$, the matrix inverse of $P$, if it exists.
\fi

\ben

\item 
\iffrancais
Determiner l'inverse de la matrice $P$, si elle existe.
\else
Determine the inverse of matrix $P$, denoted $P$.
\fi



\item
\iffrancais
On pose 
\else
Define
\fi
$D=P^{-1}AP$. 
%
%
\iffrancais
Calculer $D, D^{2}, D^{3}$ et en déduire $D^{n}$, pour tout $n$ de $\N$.
\else
Compute $D, D^{2}, D^{3}$ and deduce $D^{n}$, for all $n$ in $\N$.
\fi


\item
\iffrancais
A partir des résultats obtenus aux questions précédentes, calculer $A^{n}$,pour tout  $n$ in $\N$.
\else
From the results obtained in the previous questions, compute  $A^{n}$, for all $n$ in $\N$.
\fi


\begin{sympycode}
qs= 'zepodededzdezzpodkzepodzek'
lk = 'Matrixzedoezidjzeoidjzeodizjeoij'
\end{sympycode}


\een





\end{document}

