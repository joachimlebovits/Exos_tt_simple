::: Thm \textbf{Theoreme 1}. \emph{This statement is true, I guess.
\[\label{proto}
 2x +3 = 2.\]} :::

On peut donc dire que le théorème
\protect\hyperlink{oeizjdozij}{{[}oeizjdozij{]}} vérifie l'égalité
\protect\hyperlink{proto}{{[}proto{]}}.

This statement is true, I guess.

Exercise \#

Let \(A\) and \(P\) be the two following matrices.

\[\begin{aligned}
&A:=  \begin{pmatrix}1 & 2 & -2\\2 & 1 & -2\\2 & 2 & -3\end{pmatrix}&
& \& & 
&P:=  \begin{pmatrix}1 & 0 & 1\\1 & 1 & 1\\1 & 1 & 2\end{pmatrix}.&
\end{aligned}\]

Denote \(P^{-1}\), the matrix inverse of \(P\), if it exists.

\begin{enumerate}
\def\labelenumi{\arabic{enumi}.}
\item
  Determine the inverse of matrix \(P\), denoted \(P\).
\item
  Define \(D=P^{-1}AP\). Compute \(D, D^{2}, D^{3}\) and deduce
  \(D^{n}\), for all \(n\) in \({\mathbf{N}}\).
\item
  From the results obtained in the previous questions, compute
  \(A^{n}\), for all \(n\) in \({\mathbf{N}}\).
\end{enumerate}
