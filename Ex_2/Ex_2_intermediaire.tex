\RequirePackage{comment}
%%begin_old_tag
\begin{comment}%_1
\newif\{\%if langue == 'fr' \%\}
%commenter la ligne suivante pour l'anglais
\francaistrue %si c'est commenté alors ce sera de l'anglais. Si c'est décommenté alors tout sera en fran\c cais
%
\{\%if langue == 'fr' \%\}
\documentclass[a4paper,11pt,svgnames,openany,french]{book}
\includecomment{french}
\excludecomment{english}
\{\%elif langue == 'en' \%\}
\documentclass[a4paper,11pt,svgnames,openany,english]{book}
%\renewcommand{\partname}{Partie}
\includecomment{english}
\excludecomment{french}
\{\%endif\%\}
\input{/Users/jlebovits/Dropbox/5-Latex_et_tous_logiciels_de_math_et_terminal/1-Latex/2-Fichiers_Tex_a_inclure/Inputgene/Inputgene}
\end{comment}%_1
%%end_old_tag


%%begin_new_tag
%\begin{comment}%_2
\documentclass{book}
\usepackage[T1]{fontenc}
\usepackage[utf8]{inputenc}
\usepackage{amsmath,amssymb}
\usepackage{iftex}
\usepackage{pythontex}
\usepackage{lmodern}

\input{/Users/jlebovits/Dropbox/5-Latex_et_tous_logiciels_de_math_et_terminal/1-Latex/2-Fichiers_Tex_a_inclure/Inputgene/Inputgene_pr_pandoc}
%\end{comment}%_2
%%end_new_tag










\begin{document}


\newtheorem{Thm}{Theoreme}
%\newtheorem*{rem}{Remark}


\begin{Thm}
%\label{oeizjdozij}
This statement is true, I guess.
\begin{equation}
\label{proto}
 2x +3 = 2.
\end{equation}
\end{Thm}


On peut donc dire que le théorème \ref{oeizjdozij} vérifie l'égalité \eqref{proto}.


\begin{rem}
This statement is true, I guess.
\end{rem}



%\EX

%{\bfseries

{\Large \{\%if langue == 'fr' \%\} Exercice
\{\%elif langue == 'en' \%\} Exercise \{\%endif\%\}
}\num
%}:

\smallskip




\begin{comment}
$\N$
\end{comment}




%\begin{comment}
\{\%if langue == 'fr' \%\}
Soit $A$ et $P$ les deux matrices définies ci-dessous.
\{\%elif langue == 'en' \%\}
Let $A$ and $P$ be the two following matrices.
\{\%endif\%\}
%\end{comment}




\begin{align}
&A:=  \sympy{A}&
& \& & 
&P:=  \sympy{P}.&
\end{align}

%\{\%if langue == 'fr' \%\}
%Selon \eqref{eozrifj}, on a l'égalité
%\{\%elif langue == 'en' \%\}
%According to \eqref{eozrifj}, Equality, we clearly have 
%\{\%endif\%\}







\{\%if langue == 'fr' \%\}
On note $P^{-1}$, l'inverse de la matrice $P$, si elle existe.
\{\%elif langue == 'en' \%\}
Denote  $P^{-1}$, the matrix inverse of $P$, if it exists.
\{\%endif\%\}

\ben

\item 
\{\%if langue == 'fr' \%\}
Determiner l'inverse de la matrice $P$, si elle existe.
\{\%elif langue == 'en' \%\}
Determine the inverse of matrix $P$, denoted $P$.
\{\%endif\%\}



\item
\{\%if langue == 'fr' \%\}
On pose 
\{\%elif langue == 'en' \%\}
Define
\{\%endif\%\}
$D=P^{-1}AP$. 
%
%
\{\%if langue == 'fr' \%\}
Calculer $D, D^{2}, D^{3}$ et en déduire $D^{n}$, pour tout $n$ de $\N$.
\{\%elif langue == 'en' \%\}
Compute $D, D^{2}, D^{3}$ and deduce $D^{n}$, for all $n$ in $\N$.
\{\%endif\%\}


\item
\{\%if langue == 'fr' \%\}
A partir des résultats obtenus aux questions précédentes, calculer $A^{n}$,pour tout  $n$ in $\N$.
\{\%elif langue == 'en' \%\}
From the results obtained in the previous questions, compute  $A^{n}$, for all $n$ in $\N$.
\{\%endif\%\}





\een





\end{document}

