% Options for packages loaded elsewhere
\PassOptionsToPackage{unicode}{hyperref}
\PassOptionsToPackage{hyphens}{url}
%
\documentclass[
]{article}
\usepackage{amsmath,amssymb}
\usepackage{iftex}
\ifPDFTeX
  \usepackage[T1]{fontenc}
  \usepackage[utf8]{inputenc}
  \usepackage{textcomp} % provide euro and other symbols
\else % if luatex or xetex
  \usepackage{unicode-math} % this also loads fontspec
  \defaultfontfeatures{Scale=MatchLowercase}
  \defaultfontfeatures[\rmfamily]{Ligatures=TeX,Scale=1}
\fi
\usepackage{lmodern}
\ifPDFTeX\else
  % xetex/luatex font selection
\fi
% Use upquote if available, for straight quotes in verbatim environments
\IfFileExists{upquote.sty}{\usepackage{upquote}}{}
\IfFileExists{microtype.sty}{% use microtype if available
  \usepackage[]{microtype}
  \UseMicrotypeSet[protrusion]{basicmath} % disable protrusion for tt fonts
}{}
\makeatletter
\@ifundefined{KOMAClassName}{% if non-KOMA class
  \IfFileExists{parskip.sty}{%
    \usepackage{parskip}
  }{% else
    \setlength{\parindent}{0pt}
    \setlength{\parskip}{6pt plus 2pt minus 1pt}}
}{% if KOMA class
  \KOMAoptions{parskip=half}}
\makeatother
\usepackage{xcolor}
\setlength{\emergencystretch}{3em} % prevent overfull lines
\providecommand{\tightlist}{%
  \setlength{\itemsep}{0pt}\setlength{\parskip}{0pt}}
\setcounter{secnumdepth}{-\maxdimen} % remove section numbering
\ifLuaTeX
  \usepackage{selnolig}  % disable illegal ligatures
\fi
\IfFileExists{bookmark.sty}{\usepackage{bookmark}}{\usepackage{hyperref}}
\IfFileExists{xurl.sty}{\usepackage{xurl}}{} % add URL line breaks if available
\urlstyle{same}
\hypersetup{
  hidelinks,
  pdfcreator={LaTeX via pandoc}}

\author{}
\date{}

\begin{document}

::: Thm \textbf{Theoreme 1}. \emph{This statement is true, I guess.
\[\label{proto}
 2x +3 = 2.\]} :::

On peut donc dire que le théorème
\protect\hyperlink{oeizjdozij}{{[}oeizjdozij{]}} vérifie l'égalité
\protect\hyperlink{proto}{{[}proto{]}}.

This statement is true, I guess.

Exercise \#

Let \(A\) and \(P\) be the two following matrices.

\[\begin{aligned}
&A:=  \begin{pmatrix}1 & 2 & -2\\2 & 1 & -2\\2 & 2 & -3\end{pmatrix}&
& \& & 
&P:=  \begin{pmatrix}1 & 0 & 1\\1 & 1 & 1\\1 & 1 & 2\end{pmatrix}.&
\end{aligned}\]

Denote \(P^{-1}\), the matrix inverse of \(P\), if it exists.

\begin{enumerate}
\def\labelenumi{\arabic{enumi}.}
\item
  Determine the inverse of matrix \(P\), denoted \(P\).
\item
  Define \(D=P^{-1}AP\). Compute \(D, D^{2}, D^{3}\) and deduce
  \(D^{n}\), for all \(n\) in \({\mathbf{N}}\).
\item
  From the results obtained in the previous questions, compute
  \(A^{n}\), for all \(n\) in \({\mathbf{N}}\).
\end{enumerate}

\end{document}
