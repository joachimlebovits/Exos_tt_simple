%\RequirePackage{comment}
%%%begin_old_tag
%\begin{comment}%_1
%\newif\iffrancais
%%commenter la ligne suivante pour l'anglais
%\francaistrue %si c'est commenté alors ce sera de l'anglais. Si c'est décommenté alors tout sera en fran\c cais
%%
%\iffrancais
%\documentclass[a4paper,11pt,svgnames,openany,french]{book}
%\includecomment{french}
%\excludecomment{english}
%\else
%\documentclass[a4paper,11pt,svgnames,openany,english]{book}
%%\renewcommand{\partname}{Partie}
%\includecomment{english}
%\excludecomment{french}
%\fi
%\input{/Users/jlebovits/Dropbox/5-Latex_et_tous_logiciels_de_math_et_terminal/1-Latex/2-Fichiers_Tex_a_inclure/Inputgene/Inputgene}
%\end{comment}%_1
%%%end_old_tag


%%begin_new_tag
%\begin{comment}%_2
\documentclass{book}
\usepackage[T1]{fontenc}
\usepackage[utf8]{inputenc}
\usepackage{amsmath,amssymb}
\usepackage{iftex}
\usepackage{lmodern}

%\input{/Users/jlebovits/Dropbox/5-Latex_et_tous_logiciels_de_math_et_terminal/1-Latex/2-Fichiers_Tex_a_inclure/Inputgene/Inputgene_pr_pandoc}
%\end{comment}%_2
%%end_new_tag

\newtheorem{theo}{Theorem}









\begin{document}









\begin{equation}
 \label{A1}
 2x+1=0
\end{equation}


En utilisant la ou les solutions de l'équation \eqref{A1}, déterminer l'ensemble des solutions de l'équation suivante

\begin{equation}
 \label{A2}
 4x^{2}-1=0
\end{equation}

\eqref{A1}
est moins bien que 
\eqref{A2}
est mieux que le suivant qui n'est que  \eqref{A3}



\begin{align}
 \label{A3}
 4x^{2}-1=0
\end{align}

\eqref{A3}




\begin{theo}
 \label{theo1}
 dzeze
\end{theo}



Le theorème \ref{theo1}



\begin{align}
% \label{A4}
 1x^{2}-1=0  \label{L1A4}\\
  2x^{2}-1=0\label{LA4}\\
   3x^{2}-1=0\notag\\
    4x^{2}-1=0\label{LeA4}
\end{align}

Selon \eqref{L1A4}, on sait que: $ 1x^{2}-1=0 $. Selon \eqref{LA4}, on sait que: $   2x^{2}-1=0 $ et d'ailleurs . Selon \eqref{LeA4}, on sait que: $     4x^{2}-1=0 $. Et c'est tout.


\end{document}

