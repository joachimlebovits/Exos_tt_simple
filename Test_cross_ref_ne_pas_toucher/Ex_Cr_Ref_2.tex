\RequirePackage{comment}
%%begin_old_tag
%\begin{comment}%_1
\newif\{\%if langue == 'fr' \%\}
%commenter la ligne suivante pour l'anglais
\francaistrue %si c'est commenté alors ce sera de l'anglais. Si c'est décommenté alors tout sera en fran\c cais
%
\{\%if langue == 'fr' \%\}
\documentclass[a4paper,11pt,svgnames,openany,french]{book}
\includecomment{french}
\excludecomment{english}
\{\%elif langue == 'en' \%\}
\documentclass[a4paper,11pt,svgnames,openany,english]{book}
%\renewcommand{\partname}{Partie}
\includecomment{english}
\excludecomment{french}
\{\%endif\%\}
\input{/Users/jlebovits/Dropbox/5-Latex_et_tous_logiciels_de_math_et_terminal/1-Latex/2-Fichiers_Tex_a_inclure/Inputgene/Inputgene}
%\end{comment}%_1
%%end_old_tag


%%begin_new_tag
%\begin{comment}%_2
%\documentclass{book}
%\usepackage[T1]{fontenc}
%\usepackage[utf8]{inputenc}
%\usepackage{amsmath,amssymb}
%\usepackage{iftex}
%\usepackage{lmodern}


%\input{/Users/jlebovits/Dropbox/5-Latex_et_tous_logiciels_de_math_et_terminal/1-Latex/2-Fichiers_Tex_a_inclure/Inputgene/Inputgene_pr_pandoc}
%\end{comment}%_2
%%end_new_tag








\begin{document}



\newtheorem{Thm}{Theoreme}
\newtheorem*{rem}{Remark}


\begin{Thm}
\label{oeizjdozij}
This statement is true, I guess.
\end{Thm}



\begin{rem}
This statement is true, I guess.
\end{rem}


\begin{sympycode}
A= Matrix([[1,2,-2],[2,1,-2],[2,2,-3]]) 
P= Matrix([[1,0,1],[1,1,1],[1,1,2]])  
\end{sympycode}




\begin{align}
\label{eozrifj}
&A:=  \sympy{A}&
& \& & 
&P:=  \sympy{P}.&
\end{align}


On a donc \eqref{eozrifj} et de plus, d'après le thm \ref{oeizjdozij}.





\end{document}

