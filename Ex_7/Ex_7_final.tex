%\RequirePackage{comment}
%%%begin_old_tag
%\begin{comment}%_1
%\newif  
%\documentclass[a4paper,11pt,svgnames,openany,english]{book}
%%\renewcommand{\partname}{Partie}
%\includecomment{english}
%\excludecomment{french}
% 
%\input{/Users/jlebovits/Dropbox/5-Latex_et_tous_logiciels_de_math_et_terminal/1-Latex/2-Fichiers_Tex_a_inclure/Inputgene/Inputgene}
%\end{comment}%_1
%%%end_old_tag


%%begin_new_tag
%\begin{comment}%_2
\documentclass{book}
\usepackage[T1]{fontenc}
\usepackage[utf8]{inputenc}
\usepackage{amsmath,amssymb}
\usepackage{iftex}
\usepackage{lmodern}

\input{/Users/jlebovits/Dropbox/5-Latex_et_tous_logiciels_de_math_et_terminal/1-Latex/2-Fichiers_Tex_a_inclure/Inputgene/Inputgene_pr_pandoc}
%\end{comment}%_2
%%end_new_tag











\begin{document}









\begin{equation}
 \label{eirufheriuh}
 2x+1=0
\end{equation}


En utilisant la ou les solutions de l'équation \eqref{eirufheriuh}, déterminer l'ensemble des solutions de l'équation suivante
\begin{equation}
 \label{eirufheriuhzeodizeiçudj}
 4x^{2}-1=0
\end{equation}

\eqref{eirufheriuh}
est moins bien que 
\eqref{eirufheriuhzeodizeiçudj}
est mieux que le suivantqui n'est que  \eqref{eirufheriuhzeodizeiçudj}



\begin{align}
 \label{eirufheriuhzeodizeiçudj}
 4x^{2}-1=0
\end{align}

\eqref{eirufheriuhzeodizeiçudj}
















\end{document}

