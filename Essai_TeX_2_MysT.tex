
\documentclass{book}


\usepackage[T1]{fontenc}
\usepackage[utf8]{inputenc}
\usepackage{amsmath,amssymb}
\usepackage{iftex}
\usepackage{lmodern}



\newtheorem{theo}{Theorem}





\begin{document}









\begin{equation}
 \label{A1}
 2x+1=0
\end{equation}


En utilisant la ou les solutions de l'équation \eqref{A1}, déterminer l'ensemble des solutions de l'équation suivante

\begin{equation}
 \label{A2}
 4x^{2}-1=0
\end{equation}

\eqref{A1}
est moins bien que 
\eqref{A2}
est mieux que le suivant qui n'est que  \eqref{A3}



\begin{align}
 \label{A3}
 4x^{2}-1=0\\
  x^{2}=1/4
\end{align}

L'égalité \eqref{A3} peut se comprendre comme:




\begin{theo}
 \label{theo1}
 dzeze
\end{theo}

Le theorème \ref{theo1}





\begin{align}
 \label{A4}
 4x^{2}-1=0\\
 \label{A5}
  x^{2}=1/4
\end{align}

On a donc sur la première ligne, la ligne \eqref{A4}, et sur la deuxième ligne, la ligne \eqref{A5}.



\end{document}

